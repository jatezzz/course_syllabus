\documentclass[10pt,a4paper]{article}
\usepackage[utf8]{inputenc}
\usepackage[spanish]{babel}
\usepackage{amsmath}
\usepackage{amsfonts}
\usepackage{amssymb}
\usepackage{graphicx}
\usepackage{kpfonts}
\usepackage[table]{xcolor}
\usepackage[left=3cm,right=3cm,top=2.5cm,bottom=2.5cm]{geometry}
\author{Pablo Nepas y John Andres Trujillo}
\usepackage{tabularx} %tablas
\usepackage{booktabs} %lineas en tablas
\usepackage[T1]{fontenc}
\usepackage{calligra}
\usepackage{colortbl}
\usepackage{booktabs}
\usepackage{multirow}
\usepackage{fancyhdr}
\usepackage{pstricks}
\usepackage{soul}

\pagestyle{fancy}
\headheight=62pt
\fancyhead[L]
{	
	\begin{minipage}{2cm}
		\includegraphics[width=0.9\textwidth]{Figuras/logo.png}
	\end{minipage}}
\fancyhead[C]							
{	\begin{minipage}{10cm}
		\begin{center}
			{\bf	{UNIVERSIDAD DE LAS FUERZAS ARMADAS – ESPE\\
			VICERRECTORADO ACADÉMICO\\
			DEPARTAMENTO DE CIENCIAS EXACTAS\\
			NIVELACIÓN DE CARRERA}}	
		\end{center} 
	\end{minipage}}
\fancyhead[R]
{	
	\begin{minipage}{2cm}
		\includegraphics[width=0.9\textwidth]{Figuras/snna.png}
	\end{minipage}}

\begin{document}
{\bf Nombres:} Pablo Nepas y John Andres Trujillo\\
  {\bf Fecha:} 2 de mayo de 2014\\\\
{\bf {\blue SÍLABO DE LA ASIGNATURA:}}\\
\begin{enumerate}
	\item [1.1]	{\bf DATOS INFORMATIVOS DE LA ASIGNATURA:}\\\\
\noindent
\begin{tabularx}{\textwidth}{|X|X|X|}
\hline
\multicolumn{3}{|l|}{{\bf DEPARTAMENTO}: Ciencias Exactas}\\\hline
\multicolumn{3}{|l|}{{\bf ÁREA}: Técnica, Administrativa }\\\hline
\multicolumn{3}{|l|}{{\bf MODALIDAD}:  Presencial}\\\hline
{\bf ASIGNATURA}: Geometría y Trigonometría&{\bf CÓDIGO}&{\bf NRC}\\\hline
{\bf ÁREA DE CONOCIMIENTO}&\multicolumn{2}{l|}{Matemáticas}\\\hline
{\bf NIVEL}: Nivelación de Carrera&\multicolumn{2}{l|}{{\bf NUMERO DE CRÉDITOS:} 4}\\\hline
{\bf SESIONES/SEMANA: }&\multicolumn{2}{l|}{\bf PERÍODO ACADÉMICO: }\\\hline
{\bf PRERREQUISITOS }&\multicolumn{2}{l|}{Aprobación Examen ENES y Asignación de Cupo. }\\\hline

\end{tabularx}\\

\item [1.2.]	{\bf DATOS INFORMATIVOS DEL DOCENTE:}\\\\
\noindent
\begin{tabularx}{\textwidth}{|p{5.2cm}|X | X|}
\hline
{\bf NOMBRE DEL DOCENTE}&\\\hline
{\bf NÚMERO TELEFÓNICO}&\\\hline
{\bf CORREO ELECTRÓNICO}&\\\hline
{\bf TÍTULOS ACADÉMICOS DE TERCER  Y CUARTO NIVEL}&\\\hline

\end{tabularx}\\
\item[1.3.] {\bf DESCRIPCIÓN DE LA ASIGNATURA:} (Coordinador de área de conocimiento).\\\\
La geometría es una parte de la matemática que trata de estudiar unas idealizaciones del espacio en que vivimos, que son los puntos, las rectas y los planos, y otros elementos conceptuales derivados de ellos, como polígonos o poliedros.\\

\item[1.4.]	{\bf OBJETIVO GENERAL DE LA ASIGNATURA O MÓDULO: }\\\\
Adquirir los términos y conceptos geométricos fundamentales de la Geometría plana y  la trigonometría, que me permitan solucionar problemas cotidianos, basados en leyes, teoremas y postulados.\\

\item[1.5.]	{\bf CONTENIDOS:}\\\\
\noindent
\begin{tabularx}{\textwidth}{|p{1cm}|X|}
\hline
\centering{\bf No}&\centering \arraybackslash {\bf UNIDADES DE ESTUDIO }\\\hline
\centering\multirow{2}{*}{\bf 1}&{\bf Unidad 1:} Trigonometría\\\cline{2-2}
&Contenidos de estudios:\par\vspace{1cm}
1.	Ángulo trigonométrico: Posición estándar, coterminales, de referencia, cuadrantales, círculo trigonométrico, relaciones trigométricas de ángulos positivos y negativos, de ángulos cuadrantales.\par
2.	Triángulo rectángulo: Funciones trigonométricas de los ángulos de 45, 30 y 60 grados, Cofunciones\par
3.	Reducción de funciones.\par
4.	Gráficos de las funciones trigonométricas.  \par
5.	Gráficos de las funciones trigonométricas.\par
6.	Análisis trigonométrico: Identidades trigonométricas fundamentales. Suma y diferencia de ángulos. Ejercicios.\par
7.	Ángulos dobles, múltiples, mitad, suma a producto y de producto a suma.\par
8.	Ecuaciones Trigonométricas.   \par
9.	Ecuaciones Trigonométricas.   \\\hline
\centering\multirow{2}{*}{\bf 2}&{\bf Unidad 2:} Geometría Plana\\\cline{2-2}
&Contenidos de estudio:   \par\vspace{1cm}

1.	Proposiciones, segmentos, posiciones  relativas: punto recta, recta - recta\par
2.	Ángulos, definición, representación gráfica, elementos, denominación, medida, congruencia, clasificación, propiedades y paralelas. \par
3.	Triángulos: definición, representación gráfica, elementos, denominación, clasificación, ángulos en el triángulo.\par
4.	Triángulos: líneas, puntos notables y ángulos entre líneas fundamentales.\par
5.	Propiedades de los triángulos: Isósceles, equilátero y rectángulo.\par
6.	Triángulos: congruencia.\par
7.	Triángulos: semejanza.\par
8.	Resolución de triángulos rectángulos: relaciones métricas y trigonométricas.\par
9.	Resolución de triángulos: relaciones métricas y trigonométricas, Área.\par
10.	Ángulos: de orientación y situación.\par
11.	Círculos: Definiciones. Elementos. Ángulos en el círculo.\par
12.	Cuerdas, tangentes, secantes, propiedades. Posición relativa. \par 
13.	Longitud de arco: Circunferencia. Área del círculo. Corona, sector, segmento circular. Áreas circulares.\\\hline
\centering\multirow{2}{*}{\bf 3}&{\bf Unidad 3:} Geometría del Espacio\\\cline{2-2}
&Contenidos de estudio:   \par\vspace{1cm}

1.	Polígonos: definición, representación gráfica, elementos, denominación, clasificación, propiedades\par
2.	Cuadriláteros: clasificación, propiedades.\par
3.	Planos: posiciones relativas, ángulos diedros. ángulos poliedros. Poliedros: definición, representación gráfica, elementos, denominación, clasificación, propiedades.\par
4.	Prisma recto: definición, representación gráfica, elementos,  clasificación, propiedades.\par
5.	Cilindro de revolución: definición, representación gráfica, elementos, propiedades.\par
6.	Pirámides regulares: definición, representación gráfica, elementos, clasificación, propiedades.\par
7.	Tronco de pirámide regulare: definición, representación gráfica, elementos, clasificación, propiedades.\par
8.	Cono de revolución, esfera, volúmenes esféricos: definición, representación gráfica, elementos, propiedades.\par
\\\hline
\end{tabularx}\\

\item[1.6.]	RESULTADOS DE APRENDIZAJE: \\\\

\begin{tabularx}{\textwidth}{|X|p{1.5cm}|X|}
\hline
\centering{\bf RESULTADO DE APRENDIZAJE}&\centering{\bf NIVEL}&\centering\arraybackslash{\bf FORMA DE EVIDENCIARLO}\\\hline
Resuelve problemas de ejercicios de aplicación de ángulos.&\centering A&Lección escrita, Taller, Tareas.\\\hline
Integra información recibida para solucionar problemas trigonométricos&\centering A&Lección escrita, Taller\\\hline
Aplica conocimientos desarrollados para realizar tareas&\centering A&Tarea\\\hline
Aplica Teoremas y corolarios para la solución de ejercicios de Congruencia de Triángulos&\centering A&Lección escrita, Taller, Tareas.\\\hline
Aplica Teoremas y corolarios para la solución de ejercicios de Semejanza de Triángulos&\centering A&Lección escrita, Taller, Tareas.\\\hline
Define procesos para la solución de problemas&\centering A&Lección escrita, Taller, Tareas.\\\hline
Identifica varias alternativas de solución a un mismo problema planteado&\centering A&Lección escrita, Taller, Tareas.\\\hline
Investiga nuevas formas de demostración&\centering M&Lección escrita, Taller, Tareas.\\\hline
\end{tabularx}\\
\begin{tabularx}{\textwidth}{X p{1.5cm} X}
\centering{\bf A: alto}&\centering{\bf M: medio}&\centering\arraybackslash{\bf B: bajo}\\
\end{tabularx}\\

\item[1.7.]	{\bf METODOLOGÍA:}
\begin{enumerate}
\item[a)] Estrategias metodológicas:
\begin{itemize}
\item	Consultas puntuales podrán ser hechas al profesor mediante el uso del correo electrónico.
\item	El profesor actuará como un facilitador, por lo tanto, es su obligación diseñar estrategias y actividades de aprendizaje, que oriente a los estudiantes en qué hacer con la información científica actualizada. 
\item	Las tareas y actividades plateadas en la metodología permitirán el desarrollo de las capacidades mentales de orden superior en los estudiantes (análisis, síntesis, reflexión, pensamiento crítico, pensamiento sistémico, pensamiento creativo, manejo de información, investigación, metacognición, entre otros).\\
\end{itemize}
\item[b)]	Orientaciones metodológicas:\\
\begin{itemize}
\item	Se  diagnosticará conocimientos y habilidades adquiridas al iniciar el periodo académico.
\item	A través de preguntas y participación de los estudiantes el docente recuerda los requisitos de aprendizaje previos que permite al docente conocer cuál es la línea de base a partir del cual incorporará nuevos elementos de competencia, en caso de encontrar deficiencias enviará tareas para atender los problemas individuales.
\item	Plantear interrogantes a los estudiantes para que den sus criterios y puedan asimilar la situación problemática. 
\item	Se iniciará con explicaciones orientadoras del contenido de estudio, donde el docente plantea los aspectos más significativos, los conceptos, leyes y principios y métodos esenciales; y propone la secuencia de trabajo en cada unidad de estudio.
\item	Se buscará que el aprendizaje se base en el análisis y solución de problemas; usando información en forma significativa; favoreciendo la retención; la comprensión; el uso o aplicación de la información, los conceptos, las ideas, los principios y las habilidades en la resolución de problemas de química orgánica.
\item	Se buscará la lectura de papers para favorecer la realización de procesos de pensamiento complejo, tales como: análisis, razonamientos, síntesis, revisiones y profundización de diversos temas.
\item Se realizan prácticas de laboratorio para desarrollar las habilidades proyectadas en función de las competencias. 
\item Se realizan ejercicios orientados a la carrera y otros propios del campo de estudio.
\item	La evaluación cumplirá con las tres fases: diagnóstica, formativa y sumativa, valorando el desarrollo del estudiante en cada tarea y en especial en las evidencias del aprendizaje de cada unidad.
\end{itemize}
\end{enumerate}
\item[1.8.]		COMPORTAMIENTO ÉTICO:\\\\
El comportamiento del estudiante está sujeto al Código de Ética que tiene la Escuela, del que hay que tomar en cuenta los siguientes aspectos:
\begin{itemize}
\item	Honestidad a toda prueba (La copia de exámenes, pruebas, informes, proyectos, capítulos, ensayos, entre otros, será severamente corregida, inclusive podría ser motivo de la pérdida automática del semestre, En los trabajos se deberán incluir las citas y referencias de los autores consultados (de acuerdo a normativas aceptadas,APA, Para evitar el plagio se utilizará el programa  Plagium, Duplichecker, Víper). Si un plagio es evidenciado, podría ser motivo de la separación del curso del o los involucrados. (Código de Ética de la Universidad).
\item Respeto a la libertad de pensamiento (Respeto en las relaciones docente- alumno y alumno-alumno será exigido en todo momento, esto será de gran importancia en el desarrollo de las discusiones en clase).
\item	Orden, puntualidad y disciplina conscientes  (No se permitirá el ingreso de los estudiantes con un retraso máximo de 10 minutos, Los casos y trabajos asignados deberán ser entregados el día correspondiente).
\item Búsqueda permanente de la calidad y excelencia.
\item Igualdad de oportunidades.
\item	Respeto a las personas y los derechos humanos.
\item	Reconocimiento a la voluntad, creatividad y perseverancia.
\item	Práctica de la justicia, solidaridad y lealtad (Si es detectada la poca o ninguna participación en las actividades grupales de algún miembro de los equipos de trabajo y esto no es reportado por ellos mismos, se asumirá complicidad de ellos y serán sancionados con la nota de cero en todo el trabajo).
\item	Práctica de la verdadera amistad y camaradería.
\item	Cultivo del civismo y respeto al medio ambiente.
\item	Compromiso con la institución y la sociedad.
\item	Identidad institucional.
\item	Liderazgo y emprendimiento.
\item	Pensamiento crítico.
\item	Alta conciencia ciudadana. \\
\end{itemize}
\item[1.9.]	RECURSOS:\\\\
Aula virtual, materiales propios de la asignatura, correo electrónico. Bibliotecas virtuales–ESPE: e-libro, ProQuest, Ebrary, GALE Cengage Learning, Ebsco, IEEEXplore Digital Library, SpringerLink, Taylor \& Francis. Repositorios de tesis de grado y postgrado (Cobuec), Búsquedas avanzadas en Google y Altavista y todos los sitios que los profesores consideren confiables de acuerdo a la especialidad.
\item[1.10.]	EVALUACIÓN:\\\\
El proceso de evaluación está sujeto a la normativa establecida por el SNNA:
\begin{itemize}
\item Reglamento del SNNA
\item	Reglamento de evaluación del 13 de mayo del 2013
\item	Políticas de Aprobación, Reprobación y Anulación de Matrícula de los cursos de\\ Nivelación de Carrera del 7 de octubre del 2013.\\
\end{itemize}

%/-----------------------/Bibliografia básica/---------------------/
\noindent
\begin{tabularx}{\textwidth}{|p{0.2386\textwidth}@{}|@{}p{0.2592\textwidth}@{}|@{}p{0.1038\textwidth}@{}|@{}p{0.142\textwidth}@{}|@{}X|}\hline
\rowcolor{gray}\centering\arraybackslash {TITULO}&\centering\arraybackslash {AUTOR}&\centering\arraybackslash {ANO}&\centering\arraybackslash {IDIOMA}&\centering\arraybackslash {EDITORIAL}\\\hline
{\footnotesize Apuntes de clases de Geometría y trigonometría}&{\footnotesize \quad ABARCA, Hernán}&{\footnotesize \quad 2009}&{\footnotesize \quad Español}&{\footnotesize \quad ESPE}\\
&&&&\\\hline
{\footnotesize Geometría del espacio}&{\footnotesize \quad ABARCA, Hernán}&{\footnotesize \quad 2009}&{\footnotesize \quad Español}&{\footnotesize \quad ESPE}\\
&&&&\\\hline
{\footnotesize Geometría del espacio}&{\footnotesize \quad CALVACHE, Gonzalo}&{\footnotesize \quad 2008}&{\footnotesize \quad Español}&{\footnotesize \quad E.P.N.}\\
&&&&\\\hline
{\footnotesize Precálculo}&{\footnotesize \quad SULLIVAN Michael}&{\footnotesize \quad 2003}&{\footnotesize \quad Español}&{\footnotesize \quad Prentice Hall}\\
&&&&\\\hline
{\footnotesize Precálculo} &{\footnotesize \quad JOE GARCIA}&{\footnotesize \quad 2008}&{\footnotesize \quad Español}&{\footnotesize \quad ESPE}\\
&&&&\\
&&&&\\\hline
\end{tabularx}\\
\\
\newpage
%-------------------------------------------------------------%
\item[2.] PLANIFICACIÓN DIARIA:\\
%/-----------------------/Planificación diaria/---------------------/
\noindent
\begin{tabularx}{\textwidth}{|@{}p{0.1228\textwidth}@{}|@{}p{0.1228\textwidth}@{}|@{}p{0.1979\textwidth}@{}|@{}p{0.137\textwidth}@{}|@{}p{0.172\textwidth}@{}|@{}X@{}|}\hline
\multirow{3}{*}{\quad SESION}&\multirow{3}{*}{\quad FECHA}&\multirow{3}{*}{\quad CONTENIDO}&\multirow{3}{*}{\quad MATERIAL}&\multirow{3}{*}{\qquad TAREA}&\multirow{3}{*}{\quad RESULTADOS DE}\\
&&&&&\\
&&&&&\quad \quad APRENDIZAJE\\
&&&&&\qquad EVIDENCIAL\\\hline
&&{\footnotesize TRIGONOMETRÍA}&&&\\
&&&&&\\
&&&&&\\\hline
\multirow{15}{*}{\qquad \quad 1}&&Ángulo trigonométrico: Posición estándar, coterminales, de referencia, cuadrantales, círculo trigonométrico, relaciones trigo métricas de ángulos positivos y negativos, de ángulos cuadrantales.&&Resolver Ejercicios&Deberes, Trabajo grupal, Aula Taller\\
&&&&&\\\hline
\multirow{7}{*}{\qquad \quad 2}&&Funciones Trigonométricas para cualquier ángulo, ejercicios de aplicación&&Deberes&Aula Taller, Deberes\\
&&&&&\\\hline
\multirow{5}{*}{\qquad \quad 3}&&Funciones ángulos Agudos.  Ejercicios de aplicación&&Resolver Ejercicios&Debate, Deberes, Pruebas\\
&&&&&\\\hline
\multirow{5}{*}{\qquad \quad 4}&&Ley de senos, caso ambiguo Problemas de aplicación&&Resolver Ejercicios&Deberes, Trabajo grupal, Aula Taller\\
&&&&&\\\hline
\multirow{8}{*}{\qquad \quad 5}&&Análisis trigonométrico: Identidades trigonométricas fundamentales. Suma y diferencia de ángulos. Ejercicios.&&Resolver Ejercicios&Deberes, Consultas, lecturas\\
&&&&&\\\hline
\multirow{5}{*}{\qquad \quad 6}&&Ángulos dobles, múltiples, mitad, suma a producto y de producto a suma.&&Resolver Ejercicios&Deberes, Trabajo grupal, Aula Taller\\
&&&&&\\\hline
\multirow{5}{*}{\qquad \quad 7}&&Ángulos dobles, múltiples, mitad, suma a producto y de producto a suma.&&Deberes&Deberes, Trabajo grupal, Aula Taller\\
&&&&&\\
&&&&&\\\hline
\end{tabularx}
\begin{tabularx}{\textwidth}{|@{}p{0.1228\textwidth}@{}|@{}p{0.1228\textwidth}@{}|@{}p{0.1979\textwidth}@{}|@{}p{0.137\textwidth}@{}|@{}p{0.172\textwidth}@{}|@{}X@{}|}\hline
\multirow{5}{*}{\qquad \quad 8}&&Ecuaciones Trigonométricas.&&Resolver Ejercicios&Deberes, Trabajo grupal, Aula Taller\\
&&&&&\\
&&&&&\\\hline
\multirow{6}{*}{\qquad \quad 9}&&Análisis y gráfico de las funciones inversas.&&Resolver Ejercicios&Deberes, Trabajo grupal, Aula Taller\\
&&&&&\\
&&&&&\\\hline
&&&&&\\
&&&&&\\
&&{\footnotesize GEOMETRÍA PLANA}&&&\\\hline
&&&&Resolver Ejercicios&Deberes, Trabajo grupal, Aula Taller\\
\multirow{4}{*}{\qquad \quad 10}&&Análisis y gráfico de las funciones inversas.&&&\\\hline
&&&&Deberes&Deberes, Trabajo grupal, Aula Taller\\
\multirow{2}{*}{\qquad \quad 11}&&Segmentos, posiciones  relativas: punto recta, recta.&&&\\\hline
&&&&Resolver Ejercicios&Deberes, Trabajo grupal, Aula Taller\\
\multirow{7}{*}{\qquad \quad 12}&&Ángulos, definición, representación gráfica, elementos, denominación, medida, congruencia, clasificación, propiedades y paralelas&&&\\\hline
&&&&Resolver Ejercicios&Deberes, Trabajo grupal, Aula Taller\\
\multirow{5}{*}{\qquad \quad 13}&&Triángulos: definición, representación gráfica, elementos, denominación, clasificación, ángulos en el triángulo.&&&\\\hline
&&&&Resolver Ejercicios&Deberes, Trabajo grupal, Aula Taller\\
\multirow{5}{*}{\qquad \quad 14}&&Triángulos: líneas, puntos notables y ángulos entre líneas fundamentales.&&&\\\hline
&&&&&Cuestionario\\
\multirow{1}{*}{\qquad \quad 15}&&Evaluación 1&&&\\\hline
&&&&Resolver Ejercicios&Deberes, Trabajo grupal, Aula Taller\\
\multirow{4}{*}{\qquad \quad 16}&&Propiedades de los triángulos: Isósceles, equilátero y rectángulo.&&&\\\hline

\end{tabularx}
\begin{tabularx}{\textwidth}{|@{}p{0.1228\textwidth}@{}|@{}p{0.1228\textwidth}@{}|@{}p{0.1979\textwidth}@{}|@{}p{0.137\textwidth}@{}|@{}p{0.172\textwidth}@{}|@{}X@{}|}\hline
&&&&Deberes&Deberes, Trabajo grupal, Aula Taller\\
\multirow{1}{*}{\qquad \quad 17}&&Triángulos:\qquad \qquad congruencia. &&&\\\hline
&&&&Resolver Ejercicios&Deberes, Trabajo grupal, Aula Taller\\
\multirow{1}{*}{\qquad \quad 18}&&Triángulos: \qquad \qquad semejanza.&&&\\\hline
&&&&Resolver Ejercicios&Deberes, Trabajo grupal, Aula Taller\\
\multirow{3}{*}{\qquad \quad 19}&&Resolución de triángulos rectángulos: relaciones métricas y trigonométricas.&&&\\\hline
&&&&Resolver Ejercicios&Deberes, Trabajo grupal, Aula Taller\\
\multirow{3}{*}{\qquad \quad 20}&&Resolución de triángulos: relaciones métricas y trigonométricas, Área. &&&\\\hline
&&&&Resolver Ejercicios&Deberes, Trabajo grupal, Aula Taller\\
\multirow{1}{*}{\qquad \quad 21}&&Ángulos: de orientación y situación.&&&\\\hline
&&&&Deberes&Deberes, Trabajo grupal, Aula Taller\\
\multirow{2}{*}{\qquad \quad 22}&&Círculos: Definiciones. Elementos. Ángulos en el círculo. &&&\\\hline
&&&&Resolver Ejercicios&Deberes, Trabajo grupal, Aula Taller\\
\multirow{2}{*}{\qquad \quad 23}&&Cuerdas, tangentes, secantes, propiedades. Posición relativa.  &&&\\\hline
&&&&Resolver Ejercicios&Deberes, Trabajo grupal, Aula Taller\\
\multirow{5}{*}{\qquad \quad 24}&&Longitud de arco: Circunferencia. Área del círculo. Corona, sector, segmento circular. Áreas circulares. &&&\\
&&&&&\\\hline
\qquad \quad 25&&Evaluación 2&&Resolver Ejercicios&Evaluación\\\hline
\end{tabularx}
\begin{tabularx}{\textwidth}{|@{}p{0.1228\textwidth}@{}|@{}p{0.1228\textwidth}@{}|@{}p{0.1979\textwidth}@{}|@{}p{0.137\textwidth}@{}|@{}p{0.172\textwidth}@{}|@{}X@{}|}\hline
&&{\footnotesize GEOMÉTRIA DEL ESPACIO}&&&\\

&&&&&\\\hline
\multirow{7}{*}{\qquad \quad 26}&&Polígonos: definición, representación gráfica, elementos, denominación, clasificación, propiedades.&&Resolver Ejercicios&Deberes, Trabajo grupal, Aula Taller\\
&&&&&\\\hline
&&&&Deberes&Deberes, Consultas, lecturas, informes\\
\multirow{0}{*}{\qquad \quad 27}&&Cuadriláteros: clasificación, propiedades.&&&\\\hline
\multirow{12}{*}{\qquad \quad 28}&&Planos: posiciones relativas, ángulos diedros. Ángulos poliedros. Poliedros: definición, representación gráfica, elementos, denominación, clasificación, propiedades.&&Resolver Ejercicios&Deberes, Consultas, lecturas, informes\\
&&&&&\\\hline
\multirow{7}{*}{\qquad \quad 29}&&	Prisma recto: definición, representación gráfica, elementos,  clasificación, propiedades.&&Resolver Ejercicios&Deberes, Consultas, lecturas, informes\\
&&&&&\\\hline
\multirow{7}{*}{\qquad \quad 30}&&Cilindro de revolución: definición, representación gráfica, elementos, propiedades.&&Resolver Ejercicios&Deberes, Consultas, lecturas, informes\\
&&&&&\\\hline
\multirow{7}{*}{\qquad \quad 31}&&Pirámides regulares: definición, representación gráfica, elementos, clasificación, propiedades.&&&Deberes, Consultas, lecturas, informes\\
&&&&&\\\hline
\multirow{7}{*}{\qquad \quad 32}&&Tronco de pirámide regulare: definición, representación gráfica, elementos, clasificación, propiedades.&&&Deberes, Consultas, lecturas, informes\\
&&&&&\\\hline
\end{tabularx}
\begin{tabularx}{\textwidth}{|@{}p{0.1228\textwidth}@{}|@{}p{0.1228\textwidth}@{}|@{}p{0.1979\textwidth}@{}|@{}p{0.137\textwidth}@{}|@{}p{0.172\textwidth}@{}|@{}X@{}|}\hline
\multirow{8}{*}{\qquad \quad 33}&&Cono de revolución, esfera, volúmenes esféricos: definición, representación gráfica, elementos, propiedades.&&&Deberes, Consultas, lecturas, informes\\
&&&&&\\\hline
\multirow{1}{*}{\qquad \quad 33}&&Evaluación 3&&&Cuestionario\\
&&&&&\\\hline
\end{tabularx}\\
\end{enumerate}
\begin{flushright}
Sangolquí, 28 de abril de 2014\\
\end{flushright}
\vspace{2cm}
\begin{center}
{\bf FIRMAS DE RESPONSABILIDAD}
\end{center}
\vspace{0.05cm}
\begin{center}
\textbf{---------------------------------}
\end{center}
\begin{center}
{\bf Docente}
\end{center}
\begin{center}
\noindent
\begin{tabularx}{\textwidth}{l X r}
\textbf{---------------------------------}&&\textbf{---------------------------------}\\
{\bf ING. EDUARDO GARCÍA}&&{\bf LCDA. SONIA ROMAN}\\
{\bf {\small COORDINADOR DE ÁREA CONOCIMIENTO}} && {\bf {\small COORDINADORA ACADÉMICA NIVELACIÓN}}
\end{tabularx}
\end{center}
\end{document}